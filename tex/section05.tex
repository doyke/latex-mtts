\subsection{Definition}
\begin{frame}
	\frametitle\insertsection
	\framesubtitle\insertsubsection
	\vspace{-2em}
	\begin{itemize}
		\item {Track Initiation can start many false tracks.}
		\item {Before tracks are sent outside of the tracker, they need to be confirmed.}
		\item {The confirm process uses subsequent measurements to assess whether the track is real or false.}
		\item {The confirmation criteria must balance:}
		\begin{itemize}
			\item {Confirming false tracks with low probability,}
			\item {Rejecting real tracks with a low probability and}
			\item {Doing this in a reasonable time.}
		\end{itemize}
	\end{itemize}
\end{frame}
\begin{frame}
	\frametitle\insertsection
	\framesubtitle\insertsubsection
	\vspace{-2em}
	\textbf{Track types:} According to their different life stages, tracks can be classified into 3 cases:
	\vspace{1em}
	\begin{itemize}
		\item {\textbf{Tentative} (initiator): A track that is in the track initiation process.
			This type cannot be sure that there is sufficient evidence that it is actually a target or not.}
		\item {\textbf{Confirmed:} A track that was decided to belong to a valid target in the surveillance area. This is one end of initiation process.}
		\item \textbf{Deleted:} At the other end of the initiation process, this is a track that is decided to come from all random false alarms or a target which can no longer be detected. All of its info should be deleted.
	\end{itemize}
\end{frame}
\subsection{Track Confirmation Approaches}
\begin{frame}
	\frametitle\insertsection
	\framesubtitle\insertsubsection
	\vspace{-2em}
	\begin{itemize}
		\item {\textbf{M/N Logic}}
		\begin{itemize}
			\item {Makes a decision based on whether or not $M$ updates occur over the next $N$ opportunities.}
		\end{itemize}
		\item {\textbf{Sequential Probability Ratio Test (SPRT)}}
		\begin{itemize}
			\item {Evaluates a likelihood value based on the sequence of hits and misses.}
			\item {Likelihood is compared to thresholds to make a decision.}
			\item {the SPRT will take less time to provide the same performance as an equivalent M/N.}
		\end{itemize}
	\end{itemize}
\end{frame}
\begin{frame}
\frametitle\insertsection
	\framesubtitle\insertsubsection
	\vspace{-2em}
	\textbf {M/N Logic}
	\begin{itemize}
		\item {$M/N$ is fixed length test.}
		\item {If a track gets $M$ updated within the next $N$ opputunities, it is confirmed.}
		\item {If $N$ update opportunities pass with fewer than $M$ updates, the track is dropped.}
		\item {$M$ and $N$ are chosen to best meet design criteria.}
	\end{itemize}
\end{frame}
\begin{frame}
	\frametitle\insertsection
	\framesubtitle\insertsubsection
	\vspace{-2em}
	\begin{figure}
		\caption{\textbf{M/N Logic : Track Life Cycle Management}}
		\scalebox{0.8}{\begin{tikzpicture}[xscale=1,yscale=1,
	round/.style={circle, inner sep=0pt, fill=red!50, opacity=0.75, align=center},
	point/.style={circle, inner sep=0pt, minimum size=2mm, align=center},
	oval/.style={ellipse, draw=black, thick},
	label/.style={text width=15mm, align=center},
	node distance= 3cm and 1cm, >=stealth]
		
	%Nodes
	\node[oval, minimum height=1cm, minimum width=3cm]	(init)	at	(0,3)	{Initiated};
	\node[oval, minimum height=1cm, minimum width=3cm]	(del)	at	(4,0)	{Deleted};
	\node[oval, minimum height=1cm, minimum width=3cm]	(ten)	at	(-4,0)	{Tentative};
	\node[oval, minimum height=1cm, minimum width=3cm]	(con)	at	(0,-3)	{Confirmed};
	
	%Lines
	\path[black, thick, ->]	(init.0)	edge	[bend left=45]	node [midway, xshift=1.5cm] {false detection} (del.north);
	\path[black, thick, ->]	(del.south)	edge	[bend left=45]	node [midway, xshift=1.8cm] {consecutive misses} (con.east);
	\path[black, thick, ->]	(con.west)	edge	[bend left=45]	node [midway, xshift=-2cm] {associated plots $\geq$ M} (ten.south);
	\path[black, thick, ->]	(ten.north)	edge	[bend left=45]	node [midway, xshift=-2cm] {unassociated plots} (init.west);
	\draw[black, thick, ->]	(ten.120)	to [out=90,in=0] (-5,1.5) to [out=180,in=90] (-6,0) to [out=-90,in=180] (-5,-1.5) to [out=0,in=-90] node [midway, xshift=-3cm, yshift=2cm] {until N scans} (ten.240);
	\draw[black, thick, ->]	(ten) -- (del) node [midway, yshift=0.5cm] {associated plots $<$ M};
	
\end{tikzpicture}}
	\end{figure}
\end{frame}